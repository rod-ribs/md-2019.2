\documentclass{article}

\usepackage{bussproofs}
\usepackage{url}
\usepackage[utf8]{inputenc}
\usepackage[margin=1in]{geometry}

\begin{document}

\title{Problema dos Vestidos}
\author{Alexandre Rademaker}
\maketitle

\section{Problema}

Três irmãs – Ana, Maria e Cláudia – foram a uma festa com vestidos de
cores diferentes. Uma vestia azul, a outra branco e a Terceira
preto. Chegando à festa, o anfitrião perguntou quem era cada uma
delas. As respostas foram:

\begin{itemize}
\item A de azul respondeu: “Ana é a que está de branco”
\item A de branco falou: “Eu sou Maria”
\item A de preto disse:  “Cláudia é quem está de branco”
\end{itemize}

O anfitrião foi capaz de identificar corretamente quem era cada pessoa
considerando que:

\begin{itemize}
\item Ana sempre diz a verdade
\item Maria às vezes diz a verdade
\item Cláudia nunca diz a verdade
\end{itemize}

Pensando um pouco sobre o problema, pode-se concluir que a Ana estava
com o vestido preto, a Cláudia com o branco e a Maria com o azul.
Formalizar o problema e usar algum método dedutivo para construir um
argumento formal a favor da conclusão.  Dica: A tabela verdade teria
512 linhas!


\newpage
\section{Solução em Dedução Natural}

A prova $\Pi_1$ é a seguinte:

\begin{prooftree}\small
  \AxiomC{$(AA \lor AB) \lor AP$}

  \AxiomC{$AP$}

  \AxiomC{$AA \lor AB$}

  \AxiomC{$AA$}
  \AxiomC{$AA$}
  \AxiomC{$AA \to AB$}
  \BinaryInfC{$AB$}
  \AxiomC{$\neg AB$}
  \BinaryInfC{$\bot$}
  \UnaryInfC{$\neg AA$}
  \BinaryInfC{$\bot$}
  \UnaryInfC{$AP$}

  \AxiomC{$AB$}
  \AxiomC{$\neg AB$}
  \BinaryInfC{$\bot$}
  \UnaryInfC{$AP$}

  \TrinaryInfC{$AP$}
  \TrinaryInfC{$AP$}
\end{prooftree}

A prova $\Pi_2$ é a seguinte:

\begin{prooftree}
  \AxiomC{$\Pi_1$}
  \UnaryInfC{$AB$}
  \AxiomC{$AP \to CB$}
  \BinaryInfC{$CB$}
\end{prooftree}


A prova $\Pi_3$ é a seguinte:

\begin{prooftree}\scriptsize 
  \AxiomC{$MA \lor (MP \lor MB)$}

  \AxiomC{$MA$}

  % open 1
  \AxiomC{$MP \lor MB$}

  \AxiomC{$\Pi_1$}
  \UnaryInfC{$AP$}
  \AxiomC{$MP$}
  \AxiomC{$MP \to \neg CP \land \neg AP$}
  \BinaryInfC{$\neg CP \land \neg AP$}
  \UnaryInfC{$\neg AP$}
  \BinaryInfC{$\bot$}
  \UnaryInfC{$MA$}

  \AxiomC{$\Pi_2$}
  \UnaryInfC{$CB$}
  \AxiomC{$MB$}
  \AxiomC{$MB \to \neg CB \land \neg AB$}
  \BinaryInfC{$\neg CB \land \neg AB$}
  \UnaryInfC{$\neg CB$}
  \BinaryInfC{$\bot$}
  \UnaryInfC{$MA$}
  \TrinaryInfC{$MA$}
  % close 1

  \TrinaryInfC{$MA$}
\end{prooftree}


A resposta final é:

\begin{prooftree}
  \AxiomC{$\Pi_1$}
  \UnaryInfC{$AP$}
  \AxiomC{$\Pi_2$}
  \UnaryInfC{$CB$}
  \BinaryInfC{$AP \land CB$}
  
  \AxiomC{$\Pi_3$}
  \UnaryInfC{$MA$}
  
  \BinaryInfC{$(AP \land CB) \land MA$}
\end{prooftree}


\section{Referências}

Problema apresentado no livro `Teoria das Categorias para Ciência da
Computação' de Paulo Blauth Menezes e Edward Hermann Haeusler. Veja
\url{http://www.logicmatters.net/latex-for-logicians/} para construção
das provas.

\end{document}


%%% Local Variables:
%%% mode: latex
%%% TeX-master: t
%%% End:
